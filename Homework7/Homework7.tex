\documentclass[12pt]{article}

\usepackage{graphicx}			% Use this package to include images
\usepackage{amsmath}			% A library of many standard math expressions
\usepackage[margin=1in]{geometry}% Sets 1in margins. 
\usepackage{fancyhdr}			% Creates headers and footers
\usepackage{enumerate}          %These two package give custom labels to a list
\usepackage[shortlabels]{enumitem}
\usepackage{amsmath}
\usepackage{amssymb}
\usepackage{amsthm}

% Creates the header and footer. You can adjust the look and feel of these here.
\pagestyle{fancy}
\fancyhead[l]{Xiaoming Cao}
\fancyhead[c]{MATH521H Homework\#7}
\fancyhead[r]{\today}
\fancyfoot[c]{\thepage}
\renewcommand{\headrulewidth}{0.2pt} %Creates a horizontal line underneath the header
\setlength{\headheight}{15pt} %Sets enough space for the header

\newtheorem*{proposition}{Proposition}
\newtheorem{lemma}{Lemma}
\newtheorem{case}{Case}[]
\newtheorem*{claim}{Claim}


\setlength{\parindent}{0px}
\begin{document} %The writing for your homework should all come after this. 

\section*{Section 4.2}
\textit{Problems: 5, 6}
\subsection*{Problem 5}
\begin{enumerate}[a).]
    \item {
        $\lim_{x \rightarrow 2}(3x + 4) = 10$
        \begin{proof}
            For any $\epsilon > 0$, we let $\delta = \frac{\epsilon}{3}$, we see that 
            \begin{align*}
                |x - 2| &< \frac{\epsilon}{3} \\
                \implies |3x - 6| &< \epsilon \\
                \implies |(3x+4) - 10| &< \epsilon 
            \end{align*}
            Consequently, we are done. 
        \end{proof}
    }
    \item {
        $\lim_{x \rightarrow 0} x^3 = 0 $
        \begin{proof}
            For any $\epsilon > 0$, we let $\delta = \sqrt[3]{\epsilon}$, we see that
            \begin{align*}
                |x - 0| &< \sqrt[3]{\epsilon} \\
                \implies |x|^3 &< \epsilon \\
                \implies |x^3 - 0| &< \epsilon
            \end{align*}
            Consequently, we are done. 
        \end{proof}
    }
\end{enumerate}

\subsection*{Problem 6}
\begin{enumerate}[a).]
    \item {
        \begin{proof}
            True. Let the function be $\lim_{x \rightarrow c} f(x) = L$. 
            We see that for $\delta_s$ such that $0 < \delta_s < \delta$, we have $|x - c| < \delta_s \implies |x -c| < \delta$.
            And we already know that the orginal $\delta$ is a suitable response to the particular $\epsilon$ challenge, thus $\delta_s$ also works (since $|x - c| < \delta \implies |f(x) - L| < \epsilon$). 
        \end{proof}
    }
    \item {
        False.\\
        Consider the piece-wise function as counter-example:
        $$ 
        f(x) = \left\{ 
            \begin{array}{ll} 
                x \quad   x \ne 0\\
                1 \quad   x = 0\\
            \end{array} \right.
        $$
        We see that $\lim_{x \rightarrow 0} f(x) = 0$, but $0 \ne f(a) = 1$. 
    }
\end{enumerate}


\section*{Section 4.3}
\textit{Problems: 1, 4, 6}
\subsection*{Problem 1}
\begin{enumerate}[a).]
    \item {
        Let $\epsilon > 0$ be arbitrary, we let $\delta = \epsilon^3$, we see that 
        \begin{align*}
            |x - 0| &< \epsilon^3\\
            \implies \sqrt[3]{|x|} &< \epsilon\\
            \implies |\sqrt[3]{x} - 0| &< \epsilon\\
        \end{align*}
        Consequently, we are done. 
    }
    \item {
        Given any $\epsilon > 0$, we let $\delta = \min(2|c|, \epsilon|\sqrt[3]{c^2}|)$, we see that  $0 < (-2|c| - c)^2$, so $0 < \sqrt[3]{(-2|c| - c)^2} < \sqrt[3]{x^2} $,, and it is easy to see that $c(-2|c| - c) < (-2|c| - c)^2$, for all $c \ne 0$.
        Thus we have
        \begin{align*}
            |x - c| < \epsilon |\sqrt[3]{c^2}| \\
            \implies |\sqrt[3]{x} - \sqrt[3]{c}| = \frac{|x-c|}{|\sqrt[3]{x^2} + \sqrt[3]{cx} +\sqrt[3]{c^2}|} &< \frac{|x-c|}{|\sqrt[3]{(-2|c| - c)^2} + \sqrt[3]{c(|-2|c| - c)} +\sqrt[3]{c^2}|} < \frac{|x-c|}{|\sqrt[3]{c^2}| } < \epsilon
        \end{align*}
        Consequently, we are done.
    }
\end{enumerate}

\subsection*{Problem 4}
\begin{enumerate}[a).]
    \item {
        Example: \\
        $$ 
        f(x) = 0
        $$
        $$ 
        g(x) = \left\{ 
            \begin{array}{ll} 
                x \quad   x \ne 0\\
                1 \quad   x = 0\\
            \end{array} \right.
        $$
        We see that $\lim_{x\rightarrow 0}g(x) = 0$, but $\lim_{x\rightarrow 0} g(f(x))= 1$. 

    }
\end{enumerate}

\subsection*{Problem 6}
\begin{enumerate}[a).]
    \item {
        We see that if 
        $$
        f(x) = \left\{
            \begin{array}{ll}
                1 \quad x \ne 0 \\
                0 \quad x = 0\\
            \end{array} \right.
        $$
        $$
        g(x) = \left\{
            \begin{array}{ll}
                1 \quad x = 0 \\
                0 \quad x \ne 0\\
            \end{array} \right.
        $$
        $f(x)g(x) = 0$, a constant function, hence continous at $0$. 
        And $f(x) + g(x) = 1$, a constant function, hence continous at $0$. 
    }
    \item {
        The request is impossible. 
        We see that $f(x)$ and $f(x) + g(x)$ are continuous at 0. Thus we see that $g(x) = f(x) + g(x) - f(x)$ is continuous, hence a contraditiction ($g(x)$ is assumed to be not continous). 
    }
\end{enumerate}

\section*{Section 4.4}
\textit{Problems: 1}
\subsection*{Problem 1}
\begin{enumerate}[a).]
    \item {
        \begin{proof}
        We know that $f(x) = x$ is continuous, and we know that the production of continous funtion is continous, 
        so $g(x) = x^3 = f(x)f(x)f(x)$ is continuous.
        \end{proof}
    }
    \item {
        We let $(x_n) = n$ for $n \in \mathbb{N}$ and $(y_n) = n + \frac{1}{n}$ for all $n \in \mathbb{N}$, we see that $|y_n - x_n| = |\frac{1}{n}|$, and thus $|y_n - x_n| \rightarrow 0$. 
        We let $\epsilon_0 = 0.5$, we see that $|f(y_n) - f(x_n)| = |n^3 + 3n^2(\frac{1}{n}) + 3n(\frac{1}{n})^2 + (\frac{1}{n})^3 - n^3| > |3n| > 0.5$. 
        Consequently, it fits the sequential criterion for the absence of uniform convergence. 
    }
    \item {
        \begin{proof}
        We left $A$ be a bounded subset of the $\mathbb{R}$, we see that for all $ |a| \in A$, $a < M$  for some $M >0$, $M \in \mathbb{R}$. 
        Notice, for any $x, y \in A$, $x < M$ and $y < M$. 
        Given any $\epsilon > 0$, we let $\delta = \frac{\epsilon}{3M^2}$, we see
        \begin{align*}
            |x-y| &< \frac{\epsilon}{3M^2} \\
            3M^2|x-y| &< \epsilon \\
            3M^2|x-y| &< \epsilon \\
            |x^3 - y^3|= |(x-y)(x^3 + xy + y^2)|&<|(x-y) (M^2 + (M)(M) + M^2)| < \epsilon \\
        \end{align*}
        Consequently, we are done. 
        \end{proof}
    }
\end{enumerate}

\end{document}
