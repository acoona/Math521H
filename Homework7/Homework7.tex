\documentclass[12pt]{article}

\usepackage{graphicx}			% Use this package to include images
\usepackage{amsmath}			% A library of many standard math expressions
\usepackage[margin=1in]{geometry}% Sets 1in margins. 
\usepackage{fancyhdr}			% Creates headers and footers
\usepackage{enumerate}          %These two package give custom labels to a list
\usepackage[shortlabels]{enumitem}
\usepackage{amsmath}
\usepackage{amssymb}
\usepackage{amsthm}

% Creates the header and footer. You can adjust the look and feel of these here.
\pagestyle{fancy}
\fancyhead[l]{Xiaoming Cao}
\fancyhead[c]{MATH521H Homework\#7}
\fancyhead[r]{\today}
\fancyfoot[c]{\thepage}
\renewcommand{\headrulewidth}{0.2pt} %Creates a horizontal line underneath the header
\setlength{\headheight}{15pt} %Sets enough space for the header

\newtheorem*{proposition}{Proposition}
\newtheorem{lemma}{Lemma}
\newtheorem{case}{Case}[]
\newtheorem*{claim}{Claim}


\setlength{\parindent}{0px}
\begin{document} %The writing for your homework should all come after this. 

\section*{Section 4.2}
\textit{Problems: 5, 6}
\subsection*{Problem 5}
\begin{enumerate}[a).]
    \item {
        $\lim_{x \rightarrow 2}(3x + 4) = 10$
        \begin{proof}
            For any $\epsilon > 0$, we let $\delta = \frac{\epsilon}{3}$, we see that 
            \begin{align*}
                |x - 2| &< \frac{\epsilon}{3} \\
                \implies |3x - 6| &< \epsilon \\
                \implies |(3x+4) - 10| &< \epsilon 
            \end{align*}
            Consequently, we are done. 
        \end{proof}
    }
    \item {
        $\lim_{x \rightarrow 0} x^3 = 0 $
        \begin{proof}
            For any $\epsilon > 0$, we let $\delta = \sqrt[3]{\epsilon}$, we see that
            \begin{align*}
                |x - 0| &< \sqrt[3]{\epsilon} \\
                \implies |x|^3 &< \epsilon \\
                \implies |x^3 - 0| &< \epsilon
            \end{align*}
            Consequently, we are done. 
        \end{proof}
    }
\end{enumerate}

\subsection*{Problem 6}
\begin{enumerate}[a).]
    \item {
        \begin{proof}
            True. Let the function be $\lim_{x \rightarrow c} f(x) = L$. 
            We see that for $\delta_s$ such that $0 < \delta_s < \delta$, we have $|x - c| < \delta_s \implies |x -c| < \delta$.
            And we already know that the orginal $\delta$ is a suitable response to the particular $\epsilon$ challenge, thus $\delta_s$ also works (since $|x - c| < \delta \implies |f(x) - L| < \epsilon$). 
        \end{proof}
    }
    \item {
        False.\\
        Consider the piece-wise function as counter-example:
        $$ 
        f(x) = \left\{ 
            \begin{array}{ll} 
                x \quad   x \ne 0\\
                0 \quad   x = 0\\
            \end{array} \right.
        $$
        We see that $\lim_{x \rightarrow 0} f(x) = 1$, but $1 \ne f(a) = 0$. 
    }
\end{enumerate}


\section*{Section 4.3}
\textit{Problems: 1, 4, 6}
\subsection*{Problem 1}
\begin{enumerate}[a).]
    \item {
    }
    \item {
    }
\end{enumerate}

\subsection*{Problem 4}
\begin{enumerate}[a).]
    \item {
    }
\end{enumerate}

\subsection*{Problem 6}
\begin{enumerate}[a).]
    \item {
    }
    \item {
    }
\end{enumerate}

\section*{Section 4.4}
\textit{Problems: 1}
\subsection*{Problem 1}
\begin{enumerate}[a).]
    \item {
    }
    \item {
    }
    \item {
    }
\end{enumerate}

\end{document}
