\documentclass[12pt]{article}

\usepackage{graphicx}			% Use this package to include images
\usepackage{amsmath}			% A library of many standard math expressions
\usepackage[margin=1in]{geometry}% Sets 1in margins. 
\usepackage{fancyhdr}			% Creates headers and footers
\usepackage{enumerate}          %These two package give custom labels to a list
\usepackage[shortlabels]{enumitem}
\usepackage{amsmath}
\usepackage{amssymb}
\usepackage{amsthm}
\usepackage{mathtools}

% Creates the header and footer. You can adjust the look and feel of these here.
\pagestyle{fancy}
\fancyhead[l]{Xiaoming Cao}
\fancyhead[c]{MATH521H Homework\#9}
\fancyhead[r]{\today}
\fancyfoot[c]{\thepage}
\renewcommand{\headrulewidth}{0.2pt} %Creates a horizontal line underneath the header
\setlength{\headheight}{15pt} %Sets enough space for the header

\newtheorem*{proposition}{Proposition}
\newtheorem{lemma}{Lemma}
\newtheorem{case}{Case}[]
\newtheorem*{claim}{Claim}


\setlength{\parindent}{0px}
\begin{document} %The writing for your homework should all come after this. 

\section*{Section 5.3}
\textit{Problems: 3, 6a, 7}
\subsection*{Problem 3}
\begin{enumerate}[a).]
    \item {
        \begin{proof}
            We set $g(x) \coloneq h(x) - x$, and observe that it is continous on $[0,3]$.
            We see that $g(0) = 1$ and $g(3) = -1$. 
            Thus based on the Intermediate Value Theorem, there exists $d \in [0,3]$, such that $g(d) = 0$, which means $h(d) - d = 0$, thus $h(d) = d$. 
        \end{proof}
    }
    \item {
        \begin{proof}
            Because $h$ is a differentiable function on $[0,3]$, thus we can invoke the mean value theorem. 
            That is, there exists $c \in (0,3)$ such that $h'(c) = \frac{h(3) - h(0)}{3 - 0} = \frac{2 - 1}{3-0} = \frac{1}{3} $. 
        \end{proof}
    }
    \item {
        \begin{proof}
            Similar to part b), using the mean value theorem, we see there exists $c \in (0,1)$ such that $f'(c) = \frac{f(1) - f(0)}{ 1-0} = 1$. 
            We see $h$ is differentiable on $[1,3]$ and we see that $h(1) = 2 = h(3)$.
            Thus being fancy, we can utilise Rolle's theorem to show that there exists $d \in (1,3)$ such that $f'(d) = 0$. 
            Finally, we see that $0<\frac{1}{4} < 1$, and using Darboux's theorem, we know that there exists $L \in (c,d)$ such that $h'(L) = \frac{1}{4}$ (reminder: $(c,d) \subseteq [0,3]$). 
        \end{proof}
    }
\end{enumerate}

\subsection*{Problem 6}
\begin{enumerate}[a).]
    \item {
        \begin{proof}
            We know that for all $x \in [0,a]$, there exists $c \in (0,x)$ such that $g'(c) = \frac{g(x) - g(0)}{x-0} = \frac{g(x)}{x}$.
            And we know that $|g'(c)| \le M$ for all $c \in [0,a]$, so $|\frac{g(x)}{x}| \le M$.
            Because $x \in [0,a]$, we know that 
            \begin{align*}
                \left| \frac{g'(x)}{x} \right| &\le M \\
                \implies \left| g'(x)\right| &\le Mx
            \end{align*}
            since $x> 0$.
        \end{proof}
    }
\end{enumerate}

\subsection*{Problem 7}
\begin{proof}
    We will use a indirect proof. 
    Assume for contradiction that there exists more than one fixed points. 
    That is, there exists $a, b$, which are elements of the interval such that $a \ne b$ and $f(a) = a$ and $f(b) = b$. 
    We know that $f$ is differentiable on the interval $[a,b]$(Notice: this also implies continuity).
    Thus we can invoke the Mean Value Theorem.
    That is there exists $c \in (a,b)$ such that 
    $$f'(c) = \frac{f(b) - f(a)}{b - a} = \frac{b - a}{ b - a} = 1$$
    However, we know that $f'(x) \ne 1$.
    Consequently, we have arrived at our contradiction.
\end{proof}


\vspace*{1cm}


\section*{Section 6.2}
\textit{Problems: 2a, 8}
\subsection*{Problem 2}
\begin{enumerate}[a).]
    \item {
        \begin{claim}
            $f_n$ is continous at $0$. 
        \end{claim}
        \begin{proof}
            Given $\epsilon > 0$, we let $\delta = \frac{1}{n}$. We see that $|x-0| = |x| < \frac{1}{n} \implies |f_n(0) - f_n(x)| = |0 - 0| < \epsilon$. 
            Notice that we have chosen x in such a way such that $|x| < \frac{1}{n} \implies -\frac{1}{n} < x < \frac{1}{n}$, thus $f_n(x)$ always equals 0. \\
        \end{proof} 

        \begin{claim}
            $f$ is not continuous at 0.
        \end{claim}
        \begin{proof}
            $f(x) = \lim_{n \rightarrow \infty}f_n(x)$. We let $(a_n) = \frac{1}{n}$ for $n \in \mathbb{N}$. 
            We see that $(a_n) \rightarrow 0$, but $|f(0) - f(a_n)| = 1 > 0$.
            Thus we se $f(x_n)$ does not converge to $f(0)$, and by the Criterion for Discontinuity, we may conclude that $f$ is not continuous at $c$. \\
        \end{proof}

        \begin{claim}
            $f_n$ does not converge uniformly on $\mathbb{R}$. 
        \end{claim}
        \begin{proof}
            We will use a indirect proof. 
            Assume for contradiction that $f_n \rightarrow f$ uniformly. 
            Let $(f_n)$ be a sequence of functions that converges to $f$. 
            Using the contrapostive statement of the Continuous Limit Theorem, if $f$ is not continuous at $0$, then there exists $f_n$ is not continous at $0$. 
            However, we know that $f_n$ is continous at $0$ for all $n \in \mathbb{N}$. 
            Thus we have arrived at our contradiction.
        \end{proof}
    }
\end{enumerate}


\subsection*{Problem 8}
\begin{proof}
    We see
    $$\left| \frac{1}{g(x)} - \frac{1}{g_n(x)}\right| = \left| g_n(x) - g(x)\right| \left| \frac{1}{g(x)g_n(x)}\right| $$
    We know that $K$ is a compact set and $g(x) \ne 0$, thus we know $|g(x)|$ has a minimum $M > 0$, that is $M \le |g(x)|$, so $\frac{1}{|g(x)|} \le M$. 
    Now we will show $\frac{1}{|g_n(x)|}$ is bounded. 
    We know for all $\epsilon > 0$, there exists $N \in \mathbb{N}$ such that $|g_n(x) - g(x)| < \epsilon$ whenever $n \ge N$ and $x \in K$. 
    Thus, we choose some $\epsilon_1 < M$, and we know that 
    \begin{align*}
        |g(x) - g(x)_n|< \epsilon_1 \\
        \implies |g(x)| < \epsilon_1 + |g_n(x)| \\
        \implies |g_n(x)| > |g(x)| - \epsilon_1 \\
        \implies |g_n(x)| >  M - \epsilon_1 \\
        \implies \frac{1}{|g_n(x)|} < \frac{1}{M - \epsilon_1}\\
    \end{align*}
    let $P \coloneq \frac{1}{M(M - \epsilon_1)}$, we see that 
    $$\frac{1}{|g(x)g_(x)|} < \frac{1}{M(M - \epsilon_1)} = P$$
    Consequently, $\left| \frac{1}{g(x)} - \frac{1}{g_n(x)}\right| = \left| g(x) - g_n(x)\right| \left| \frac{1}{g(x)g_n(x)}\right| $.
    And so for any $\epsilon > 0$, we can choose an $N \in \mathbb{N}$ such that $ n \ge N \implies |g(x) - g_n(x)| < \frac{\epsilon}{P} \implies \left| \frac{1}{g(x)} - \frac{1}{g_n(x)}\right| < \epsilon$, and we got our desired result.

    
    
\end{proof}


\end{document}
