
\documentclass[12pt]{article}

\usepackage{graphicx}			% Use this package to include images
\usepackage{amsmath}			% A library of many standard math expressions
\usepackage[margin=1in]{geometry}% Sets 1in margins. 
\usepackage{fancyhdr}			% Creates headers and footers
\usepackage{enumerate}          %These two package give custom labels to a list
\usepackage[shortlabels]{enumitem}
\usepackage{amsmath}
\usepackage{amssymb}
\usepackage{amsthm}

% Creates the header and footer. You can adjust the look and feel of these here.
\pagestyle{fancy}
\fancyhead[l]{Xiaoming Cao}
\fancyhead[c]{MA521H Homework\#1}
\fancyhead[r]{\today}
\fancyfoot[c]{\thepage}
\renewcommand{\headrulewidth}{0.2pt} %Creates a horizontal line underneath the header
\setlength{\headheight}{15pt} %Sets enough space for the header

\newtheorem*{proposition}{Proposition}
\newtheorem{lemma}{Lemma}
\newtheorem{Case}{Case}[]


\setlength{\parindent}{0px}
\begin{document} %The writing for your homework should all come after this. 

\section*{Section 1.2}
\textit{Problems: 3ab, 8, 10ac }

\subsection*{Problem 3}
\begin{enumerate}[a).]
    \item {
    False.
    We define $A_n = \{n, n+1, n+2, \dots\}$ for all $n \in \mathbb{N}$. 
    We see that $A_1 \supseteq A_2 \supseteq A_3 \dots$, however $\bigcap_{n=1}^{\infty}A_n = \emptyset$, thus is not infinite.
    The proof below would establish that $\bigcap_{n=1}^{\infty}A_n$ is indeed empty. 

    \begin{proof}
        We will use a proof by contradiction. 
        Assume $ a \in \bigcap_{n=1}^{\infty}A_n$, we know that $a \in A_n$ for all $n \in \mathbb{N}$.
        But wee see that $A_{a+1} = \{a+1, a+2, a+3, \dots\}$, and that $a \notin A_{a+1}$. 
        Thus we have arrived at our contradiction.
    \end{proof}
    }
    \item {
    True.
    }

\end{enumerate}

\subsection*{Problem 8}

\begin{enumerate}[a).]
    \item {
        Example: $f: \mathbb{N} \rightarrow \mathbb{N}$, $f(x) = x+1$.
        It is not surjective since no element of the domain maps to 1 in the co-domain. 
    }
    \item {
    Example: $f: \mathbb{N} \rightarrow \mathbb{N}$, $f(x) = \lceil  \log{x} \rceil$
    }
    \item {
    Example: $f: \mathbb{N} \rightarrow \mathbb{Z}$, $f(x) = (-1)^{x}{\lceil \frac{x}{2}\rceil}$
    }
\end{enumerate}

\subsection*{Problem 10}
\begin{enumerate}[a).]
    \item {
        False.
        If $a = b$, we see that $a = b < b + \epsilon$ for some $\epsilon > 0$. 
        However, we see that $a \nless b$, and thus the statement is false. 
    }

\end{enumerate}

\begin{enumerate}[c).]
    \item {
    True. 
    \begin{proof}
        We will first establish the forward direction of the proof. 
        We know that $b < b + \epsilon$ for all $\epsilon > 0$, and so $a \le b < b + \epsilon$. 
        Consequently, $a < b + \epsilon$. 
        Now we will show the converse of this statement using contradiction. 
        That is, we assume $a < b + \epsilon$ for all $\epsilon > 0$ and $a>b$. 
        Because $a > b$, we know that there exists some $r \in \mathbb{R}$ such that $r = a - b$. 
        We see that $a > b + \frac{r}{2}$ as $\frac{r}{2} < r$.
        We now arrived at our contraction because we know that $a < b + \epsilon$ for all $\epsilon > 0$.
    \end{proof}
    }
\end{enumerate}


\vspace*{1cm}


\section*{Section 1.3}
\textit{Problems: 2, 6, 8}

\subsection*{Problem 2}
\begin{enumerate}[a).]
    \item {
    A = \{12\}, inf B = 12 = sup B
    }
    \item {
    It is not possible, as in a finite set, the largest element will be the supremum. 
    }
    \item {
    $A = \{ x \; \vert \; \pi < x < 5 \}$
    }
\end{enumerate}

\subsection*{Problem 6}

\begin{enumerate}[a).]
    \item {
        \begin{proof}
            If $c \in A + B$, we know that $c = a + b$ for some $a \in A$ and $b \in B$. 
            Because $s = supA \ge a$ for all $a \in A$, and $t = supB \ge b$ for all $b \in B$, therefore $s+t \ge c$ for all $c \in A + B$. 
            And by definition, $s+t$ is an uppper bound for $A + B$. 
        \end{proof}
    }
    \item {
        \begin{proof}
            We will use a proof by contradiction. 
            We assume that for all $u$ that is an upper bound for $A + B$, $a \in A$, and $t > u - a$. 
            Thus we see $t+a > u$ for all upper bound $u$.
            We know that $t + a \le t + s$ since $s = supA \ge a$, and from (a). we know that $s+t$ is an upper bound. 
            We let $u = s + t + 1$, we know that $u > s+t \ge t + a$. 
            However, we assumed that $u < a+t$, hence we have arrived at our contradiction.
        \end{proof}

    }
    \item {
        \begin{proof}
            We let $u = sup(A+B)$.
            From (b). we know that $t \le u -a$ and $s \le u -b$. 
            Therefore $t+s \le 2u - a - b$, and so $t+s + (a+b) \le 2u$.
            We see $t + s + (a + b) \le t + s + u \le 2u$.
            Consequently, $s + t \le u$, and establishing the fact that $Sup(A) + Sup(B) = Sup(A+B)$. 
        \end{proof}
    }
    \item{
        \begin{proof}
            We know that $s - \epsilon < a$ for some $a \in A$ and $\epsilon > 0$, and $t - \epsilon < b$ for some $b \in B$ and $\epsilon > 0$.
            Thus, we see $s + t - 2\epsilon < a + b$, and further more that $(a+b) \in A + B$.
            We let $\epsilon_1 \equiv 2\epsilon$, and $c \equiv a + b$. 
            And we see that $s+t - \epsilon_1 < c$ for all $c \in A + B$ and $\epsilon_1 > 0$.
            Consequently, by Lemma 1.3.8, this is equivalent to $s + t = sup(A+B)$.
        \end{proof}
    }
\end{enumerate}

\subsection*{Problem 8}
\begin{enumerate}[a).]
    \item {
        suprema: 1\\
        infima: 0
    }
    \item {
        suprema: 1\\
        infima: -1 
    }
    \item {
        suprema: $\frac{1}{3}$\\
        infima: $\frac{1}{4}$
    }
    \item {
        suprema: 1\\
        infima: 0
    }
\end{enumerate}


\end{document}
