
\documentclass[12pt]{article}

\usepackage{graphicx}			% Use this package to include images
\usepackage{amsmath}			% A library of many standard math expressions
\usepackage[margin=1in]{geometry}% Sets 1in margins. 
\usepackage{fancyhdr}			% Creates headers and footers
\usepackage{enumerate}          %These two package give custom labels to a list
\usepackage[shortlabels]{enumitem}
\usepackage{amsmath}
\usepackage{amssymb}
\usepackage{amsthm}

% Creates the header and footer. You can adjust the look and feel of these here.
\pagestyle{fancy}
\fancyhead[l]{Xiaoming Cao}
\fancyhead[c]{MA521H Homework\#1}
\fancyhead[r]{\today}
\fancyfoot[c]{\thepage}
\renewcommand{\headrulewidth}{0.2pt} %Creates a horizontal line underneath the header
\setlength{\headheight}{15pt} %Sets enough space for the header

\newtheorem*{proposition}{Proposition}
\newtheorem{lemma}{Lemma}
\newtheorem{Case}{Case}[]


\setlength{\parindent}{0px}
\begin{document} %The writing for your homework should all come after this. 

\section*{Section 1.2}
\textit{Problems: 3ab, 8, 10ac }

\subsection*{Problem 3}
\begin{enumerate}[a).]
    \item {
    False.
    We define $A_n = \{n, n+1, n+2, \dots\}$ for all $n \in \mathbb{N}$. 
    We see that $A_1 \supseteq A_2 \supseteq A_3 \dots$, however $\bigcap_{n=1}^{\infty}A_n = \emptyset$, thus is not infinite.
    the proof below would establish that $\bigcap_{n=1}^{\infty}A_n$ is indeed empty. 

    \begin{proof}
        We will use a proof by contradiction. 
        Assume $ a \in \bigcap_{n=1}^{\infty}A_n$, we know that $a \in A_n$ for all $n \in \mathbb{N}$.
        But wee see that $A_{a+1} = \{a+1, a+2, a+3, \dots\}$, and that $a \notin A_{a+1}$. 
        Thus we have arrived at our contradiction.
    \end{proof}
    }
    \item {
    True.
    }

\end{enumerate}

\subsection*{Problem 8}

\begin{enumerate}[a).]
    \item {
        Example: $f: \mathbb{N} \rightarrow \mathbb{N}$, $f(x) = x+1$.
        It is not surjective since no element of the domain maps to 1 in the co-domain. 
    }
    \item {
    Example: $f: \mathbb{N} \rightarrow \mathbb{N}$, $f(x) = \lceil  \log{x} \rceil$
    }
    \item {
    Example: $f: \mathbb{N} \rightarrow \mathbb{Z}$, $f(x) = (-1)^{x}{\lceil \frac{x}{2}\rceil}$
    }
\end{enumerate}

\subsection*{Problem 10}
\begin{enumerate}[a).]
    \item {
        False.
        If $a = b$, we see that $a = b < b + \epsilon$ for some $\epsilon > 0$. 
        However, we see that $a \nless b$, and thus the statement is false. 
    }

\end{enumerate}

\begin{enumerate}[c).]
    \item {
    True. 
    \begin{proof}
        We will first establish the forward direction of the proof. 
        We know that $b < b + \epsilon$ for all $\epsilon > 0$, and so $a \le b < b + \epsilon$. 
        Consequently, $a < b + \epsilon$. 
        Now we will show the converse of this statement using contradiction. 
        That is, we assume $a < b + \epsilon$ for all $\epsilon > 0$ and $a>b$. 
        Because $a > b$, we know that there exists some $r \in \mathbb{R}$ such that $r = a - b$. 
        We see that $a > b + \frac{r}{2}$ as $\frac{r}{2} < r$.
        We now arrived at our contraction because we know that $a < b + \epsilon$ for all $\epsilon > 0$.
    \end{proof}
    }
\end{enumerate}


\vspace*{1cm}


\section*{Section 1.3}
\textit{Problems: 2, 6, 8}

\subsection*{Problem 2}
\begin{enumerate}[a).]
    \item {

    }
    \item {

    }
    \item {

    }
\end{enumerate}

\subsection*{Problem 6}

\begin{enumerate}[a).]
    \item {
    
    }
    \item {

    }
    \item {

    }
    \item{

    }
\end{enumerate}

\subsection*{Problem 8}
\begin{enumerate}[a).]
    \item {

    }
    \item {

    }
    \item {

    }
    \item{

    }
\end{enumerate}


\end{document}
