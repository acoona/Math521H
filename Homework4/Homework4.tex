\documentclass[12pt]{article}

\usepackage{graphicx}			% Use this package to include images
\usepackage{amsmath}			% A library of many standard math expressions
\usepackage[margin=1in]{geometry}% Sets 1in margins. 
\usepackage{fancyhdr}			% Creates headers and footers
\usepackage{enumerate}          %These two package give custom labels to a list
\usepackage[shortlabels]{enumitem}
\usepackage{amsmath}
\usepackage{amssymb}
\usepackage{amsthm}

% Creates the header and footer. You can adjust the look and feel of these here.
\pagestyle{fancy}
\fancyhead[l]{Xiaoming Cao}
\fancyhead[c]{MATH521H Homework\#4}
\fancyhead[r]{\today}
\fancyfoot[c]{\thepage}
\renewcommand{\headrulewidth}{0.2pt} %Creates a horizontal line underneath the header
\setlength{\headheight}{15pt} %Sets enough space for the header

\newtheorem*{proposition}{Proposition}
\newtheorem{lemma}{Lemma}
\newtheorem{case}{Case}[]
\newtheorem*{claim}{Claim}


\setlength{\parindent}{0px}
\begin{document} %The writing for your homework should all come after this. 

\section*{Section 2.3}
\textit{Problems: 1a, 5, 7ab}
\subsection*{Problem 1}
\begin{enumerate}[a).]
    \item {
    \begin{proof}
        Because for all $\epsilon > 0$, $|x_n| < \epsilon$ and $\epsilon^2 > 0$, we let $|x_n| < \epsilon^2$.
        We see that $x_n < \epsilon^2$ since $x_n \ge 0$. And consequently $\sqrt{x_n} < \epsilon$ for all $\epsilon > 0$. 
    \end{proof}
    }
\end{enumerate}

\subsection*{Problem 5}
        \begin{proof}
            (forwards direction) We want to show $(z_n)$ is convergent if $(y_n)$ and $(x_n)$ are both convergent with $\lim_{}x_n = \lim_{}y_n$. 
            Notice that $x_n = (z_1, z_3, z_5, \dots, z_{2n-1})$, which is the odd numbered terms of $z_n$. 
            And we see that $y_n = (z_2, z_3, z_4, \dots, z_{2n})$, which is the even numbered terms of $z_n$. 
            For all $\epsilon > 0$, there exists $N_1 \in \mathbb{N}$, such that $ n \ge N_1 \implies |x_n - c| = |z_{2n-1} - c|< \epsilon$ for some $c$. 
            And for all $\epsilon > 0$, there exists $N_2 \in \mathbb{N}$, such that $ n \ge N_2 \implies |y_n - c| = |z_{2n} - c| < \epsilon$.
            We let $N = \max{(N_1, N_2)}$, and we know that both the even and odd terms of $z_n$ converges, conseqeuntly, $z_n$ is convergent.\\

            (backwards direction) We want to show that $(x_n)$ and $(y_n)$ are both convergent with $\lim{}x_n = \lim{}y_n$.
            We see that $(x_n) = (z_1, z_3, z_5, \dots, z_{2n-1})$ and $(y_n) = (z_2, z_4, z_6, \dots, z_{2n})$. 
            We know for all $\epsilon > 0$, there exists $N \in \mathbb{N}$ such that $n \ge N \implies |z_n - c| < \epsilon$.
            We know that $2n-1 \ge n$ and $x = z_{2n-1}$, conseqeuntly, $|x_n - c| < \epsilon$. 
            Same reasoning, ${2n} > n$, and thus $|y_n - c| < \epsilon$. 
        \end{proof}

\subsection*{Problem 7}
\begin{enumerate}[a).]
    \item {
    Let $(x_n) = (1, -1, 1, \dots, (-1)^{n-1})$ for $n \in \mathbb{N}$ and $(y_n) = (-1, 1, -1, \dots. (-1)^{n})$.
    We see that $x_n$ and $y_n$ both diverges, and $(x_n + y_n) = (0,0,0,\dots)$ converges. 
    }

    \item {
    Because $(x_n)$ and $(x_n + y_n)$ converges. 
    We see that $(-x_n)$ converges, and thus $(x_n + y_n) + (-x_n) = (y_n)$ converges.
    However, it was given that $(y_n)$ diverges, consequently, the request is impossible by referencing the proper theorems. 
    }
\end{enumerate}




\vspace*{1cm}

\section*{Section 2.4}
\textit{Problems: 1abc, 2a}
\subsection*{Problem 1}
\begin{enumerate}[a).]
    \item {
        \begin{proof}
        We can show that the sequence is Monotone convergent by showing that it is bounded and monotone.
        We will use induction to show that the sequence is monotone. 
        We see that $x_1 = 3$ and $x_2 = \frac{1}{4 - 3} = 1$, and $3 > 1$. 
        We assume that $x_k < x_{k-1}$ and we need to show that $x_{k+1} < x_{k}$. 
        We see 
        \begin{align*}
            x_{k-1} &> x_{k}\\
            \implies - x_{k-1} &< -x_{k}\\
            \implies4 - x_{k-1} &< 4 -x_{k}\\
            \implies \frac{1}{4 - x_{k-1}} &> \frac{1}{4 -x_{k}}\\
            \implies x_k &> x_{k+1}
        \end{align*}
        We have thus showed that $(x_n)$ is decreasing for all $n \in \mathbb{N}$. 
        Now we have to show that $(x_n)$ is bounded.
        Because all terms are strictly decreasing, therefore no term can be greater than $x_1 = 3$. 
        Notice that $4 - x_n > 0$ if $x_n < 4$, and we know that $x_1 = 3 < 4$, so conseqeuntly, $\frac{1}{4 - x_n} > 0$ for all $n \in \mathbb{N}$ (can be easily proven using induction). 
        conseqeuntly, $4 > x_n > 0$ and we see that $|x_n| < 4$ for all $n \in \mathbb{N}$. 
        Then, by the Montone convergent theorem, we know that the sequence defined by $x_1 = 3$ and $x_{n+1} = \frac{1}{4 - x_n}$ is convergent. 
        \end{proof}

    }
    \item {
        \begin{proof}
        We know that $\lim_{} x_n$ exists, let it be $x$. 
        Therefore, we know that for all $\epsilon > 0$, there exists a $N \in \mathbb{N}$, such that $ n \ge N \implies |x_n - x| < \epsilon$. 
        We know that $n+1 > n \ge N$, so conseqeuntly, $|x_{n+1} - x| < \epsilon$. 
        And thus, $\lim_{} x_{n+1}$ exists and equals to the same value.
        \end{proof}

    }
    \item {
        We see 
        \begin{align*}
            x_{n+1} &= \frac{1}{4 - x_n}\\
            \lim_{}x_{n+1} &= \lim_{} \left( \frac{1}{4 - x_n} \right)\\
            \lim_{}x_{n+1} &=  \frac{\lim_{}(1)}{\lim_{}(4 - x_n)} \\
            \lim_{}x_{n+1} &=  \frac{\lim_{}(1)}{\lim_{}(4 - x_n)} \\
        \end{align*}
            Because $\lim_{}x_n = \lim_{}x_{n+1}$, we let it be $x$. 
            We have 
        \begin{align*}
            x &= \frac{1}{4-x}\\
            x(4-x) &= 1\\
            4x-x^2 &= 1\\
            x^2 -4x +1 &= 0\\
            x^2 -4x +4 &= 3\\
            (x-2)^2 &= 3\\
            x &= 2 \pm \sqrt{3}\\
        \end{align*}
        We know that $x_n < 3$ and is decreasing, therefore $\lim_{}x_n = 2 - \sqrt{3}$. 
    }
\end{enumerate}


\subsection*{Problem 2}
\begin{enumerate}[a).]
    \item {
        We see that we can express $(y_n)$ explicitly as $(x_n) = (1,2,1,2,1,\dots)$, and it does not converge.
        Therefore this argument is wrong because both $y_{n+1}$ and $y_n$ does not converge to a value.

    }
\end{enumerate}

\vspace*{1cm}

\section*{Section 2.5}
\textit{Problems: 1ab}
\subsection*{Problem 1}
\begin{enumerate}[a).]
    \item {
    The request is impossible, as we know that the subsequence of the subsequence is also a subsequence of the original sequence. 
    And because the subsequence is bounded, therefore via the Bolzano-Weierstrass Theorem, we know that the subsequence contains a convergent subsequence, therefore the orignal sequence also contains that convergent subsequence.
    }
    \item {
    By definition, we know that that a sequence is a funciton whose domain is the natural numbers.
    Thus we let 
        $$ 
        f(x) = \left\{ 
            \begin{array}{ll} 
                \frac{1}{n} \quad \text{n is odd}, n \in \mathbb{N}\\
                1 - \frac{1}{n} \quad \text{n is even}, n \in \mathbb{N} \\
            \end{array} \right.
        $$
    We see $a_n = (1,\frac{1}{2},\frac{1}{3}, \frac{3}{4}, \dots)$.
    the subsequence, namly $a_{n_k}$ where $n_k = 2k - 1$ for $k \in \mathbb{N}$ converges to 0.
    The subsequence of $a_{n_k}$ where $n_k = 2k $ for $k \in \mathbb{N}$ converges to 1. 
    And in addition, we know that the original sequence diverges due to the Divergance criteria. 
    }
\end{enumerate}

\end{document}
