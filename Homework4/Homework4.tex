\documentclass[12pt]{article}

\usepackage{graphicx}			% Use this package to include images
\usepackage{amsmath}			% A library of many standard math expressions
\usepackage[margin=1in]{geometry}% Sets 1in margins. 
\usepackage{fancyhdr}			% Creates headers and footers
\usepackage{enumerate}          %These two package give custom labels to a list
\usepackage[shortlabels]{enumitem}
\usepackage{amsmath}
\usepackage{amssymb}
\usepackage{amsthm}

% Creates the header and footer. You can adjust the look and feel of these here.
\pagestyle{fancy}
\fancyhead[l]{Xiaoming Cao}
\fancyhead[c]{MATH521H Homework\#4}
\fancyhead[r]{\today}
\fancyfoot[c]{\thepage}
\renewcommand{\headrulewidth}{0.2pt} %Creates a horizontal line underneath the header
\setlength{\headheight}{15pt} %Sets enough space for the header

\newtheorem*{proposition}{Proposition}
\newtheorem{lemma}{Lemma}
\newtheorem{case}{Case}[]
\newtheorem*{claim}{Claim}


\setlength{\parindent}{0px}
\begin{document} %The writing for your homework should all come after this. 

\section*{Section 2.3}
\textit{Problems: 1a, 5, 7ab}
\subsection*{Problem 1}
\begin{enumerate}[a).]
    \item {
    \begin{proof}
        Because for all $\epsilon > 0$, $|x_n| < \epsilon$ and $\epsilon^2 > 0$, we let $|x_n| < \epsilon^2$.
        We see that $x_n < \epsilon^2$ since $x_n \ge 0$. And consequently $\sqrt{x_n} < \epsilon$. 
    \end{proof}
    }
\end{enumerate}

\subsection*{Problem 5}
        \begin{proof}
            (forwards direction) We want to show $(z_n)$ is convergent if $(y_n)$ and $(x_n)$ are both convergent with $\lim_{}x_n = \lim_{}y_n$. 
            Notice that $x_n = (z_1, z_3, z_5, \dots, z_{2n-1})$, which is the odd numbered terms of $z_n$. 
            And we see that $y_n = (z_2, z_3, z_4, \dots, z_{2n})$, which is the even numbered terms of $z_n$. 
            For all $\epsilon > 0$, there exists $N_1 \in \mathbb{N}$, such that $ n \ge N_1 \implies |x_n - c| = |z_{2n-1} - c|< \epsilon$ for some $c$. 
            And for all $\epsilon > 0$, there exists $N_2 \in \mathbb{N}$, such that $ n \ge N_2 \implies |y_n - c| = |z_{2n} - c| < \epsilon$.
            We let $N = \max{(N_1, N_2)}$, and we know that both the even and odd terms of $z_n$ converges, conseqeuntly, $z_n$ is convergent.\\

            (backwards direction) We want to show that $(x_n)$ and $(y_n)$ are both convergent with $\lim{}x_n = \lim{}y_n$.
            We see that $(x_n) = (z_1, z_3, z_5, \dots, z_{2n-1})$ and $(y_n) = (z_2, z_4, z_6, \dots, z_{2n})$. 
            We know for all $\epsilon > 0$, there exists $N \in \mathbb{N}$ such that $n \ge N \implies |z_n - c| < \epsilon$.
            We know that $x_n = z_{2n-1} \ge z_{n}$, conseqeuntly, $|x_n - c| < \epsilon$. 
            Same reasoning, $y_n = z_{2n} > z_n$, and thus $|y_n - c| < \epsilon$. 
        \end{proof}

\subsection*{Problem 7}
\begin{enumerate}[a).]
    \item {
    Let $(x_n) = (1, -1, 1, \dots, (-1)^{n-1})$ for $n \in \mathbb{N}$ and $(y_n) = (-1, 1, -1, \dots. (-1)^{n})$.
    We see that $x_n$ and $y_n$ both converges, and $(x_n + y_n) = (0,0,0,\dots)$ diverges. 
    }

    \item {
    Because $(x_n)$ and $(x_n + y_n)$ converges. 
    We know that $(-x_n)$ converges, and thus $(x_n + y_n) + (-x_n) = (y_n)$ converges.
    However, it was given that $(y_n)$ diverges, consequently, the request is impossible by referencing the proper theorems. 
    }
\end{enumerate}




\vspace*{1cm}

\section*{Section 2.4}
\textit{Problems: 1abc, 2a}
\subsection*{Problem 1}
\begin{enumerate}[a).]
    \item {
    }
    \item {

    }
    \item {

    }
\end{enumerate}
\subsection*{Problem 2}
\begin{enumerate}[a).]
    \item {

    }
\end{enumerate}

\vspace*{1cm}

\section*{Section 2.5}
\textit{Problems: 1ab}
\subsection*{Problem 1}
\begin{enumerate}[a).]
    \item {

    }
    \item {

    }
\end{enumerate}

\end{document}
