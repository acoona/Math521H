\documentclass[12pt]{article}

\usepackage{graphicx}			% Use this package to include images
\usepackage{amsmath}			% A library of many standard math expressions
\usepackage[margin=1in]{geometry}% Sets 1in margins. 
\usepackage{fancyhdr}			% Creates headers and footers
\usepackage{enumerate}          %These two package give custom labels to a list
\usepackage[shortlabels]{enumitem}
\usepackage{amsmath}
\usepackage{amssymb}
\usepackage{amsthm}
\usepackage{mathtools}

% Creates the header and footer. You can adjust the look and feel of these here.
\pagestyle{fancy}
\fancyhead[l]{Xiaoming Cao}
\fancyhead[c]{MATH521H Homework\#11}
\fancyhead[r]{\today}
\fancyfoot[c]{\thepage}
\renewcommand{\headrulewidth}{0.2pt} %Creates a horizontal line underneath the header
\setlength{\headheight}{15pt} %Sets enough space for the header

\newtheorem*{proposition}{Proposition}
\newtheorem{lemma}{Lemma}
\newtheorem{case}{Case}[]
\newtheorem*{claim}{Claim}


\setlength{\parindent}{0px}
\begin{document} %The writing for your homework should all come after this. 

\section*{Section 6.5}
\textit{Problems: 1a, 6}

\subsection*{Problem 3}
\begin{enumerate}[a).]
    \item {
    }
\end{enumerate}

\subsection*{Problem 6}


\vspace*{1cm}


\section*{Section 6.6}
\textit{Problems: 2a}

\subsection*{Problem 2}
\begin{enumerate}[a).]
    \item {
    }
\end{enumerate}


\vspace*{1cm}


\section*{Section 7.2}
\textit{Problems: 4}

\subsection*{Problem 4}
We know that there exists a partition $P$ with $L(g, P) = U(g,P)$. 
$P$ can be expressed as ${x_0, x_1, \dots, x_n}$, and we see that 
\begin{align*}
&\sum_{k=1}^{n}m_k(x_k - x_{k-1}) = \sum_{k=1}^{n}M_k(x_k - x_{k-1}) \\
\implies &\sum_{k=1}^{n}(M_k - m_k)(x_k - x_{k-1}) = 0\\
\end{align*}
We know that $x_{k} - x_{k-1} \ne 0$, thus $M_k = m_k$.
This indicates that $\sup \{g(x): x \in [x_{k-1}, x_{k}]\}$ = $\inf \{g(x): x \in [x_{k-1}, x_{k}]\}$. 
We can easily see that $f(x)$ is a constant over the interval $[x_{k-1}, x_k]$ (A rather simple proof using contradiciton), and $g(x_{k-1}) = g(x_k)$.
And thus we see $g(x_0) = g(x_1) = \dots = g(x_n)$ (via induction), and so we know that $g(x)$ is constant over the entire interval [a,b]. 
This means it is integrable (constant implies continuity implies integrability). 
And the value of $\int_{a}^{b}g = (b-a)g(c)$ for any $c \in [a,b]$. 






\end{document}
