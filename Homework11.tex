\documentclass[12pt]{article}

\usepackage{graphicx}			% Use this package to include images
\usepackage{amsmath}			% A library of many standard math expressions
\usepackage[margin=1in]{geometry}% Sets 1in margins. 
\usepackage{fancyhdr}			% Creates headers and footers
\usepackage{enumerate}          %These two package give custom labels to a list
\usepackage[shortlabels]{enumitem}
\usepackage{amsmath}
\usepackage{amssymb}
\usepackage{amsthm}
\usepackage{mathtools}

% Creates the header and footer. You can adjust the look and feel of these here.
\pagestyle{fancy}
\fancyhead[l]{Xiaoming Cao}
\fancyhead[c]{MATH521H Homework\#11}
\fancyhead[r]{\today}
\fancyfoot[c]{\thepage}
\renewcommand{\headrulewidth}{0.2pt} %Creates a horizontal line underneath the header
\setlength{\headheight}{15pt} %Sets enough space for the header

\newtheorem*{proposition}{Proposition}
\newtheorem{lemma}{Lemma}
\newtheorem{case}{Case}[]
\newtheorem*{claim}{Claim}


\setlength{\parindent}{0px}
\begin{document} %The writing for your homework should all come after this. 

\section*{Section 6.5}
\textit{Problems: 1a, 6}

\subsection*{Problem 1}
\begin{enumerate}[a).]
    \item {
        We see that $g(x) = \sum_{n = 1}^{\infty}(-1)^{n-1}\frac{x^n}{n}$, and thus $|g(x)| = \sum_{n=1}^{\infty}\frac{x^n}{n}$. 
        We know that the geometric series $\sum_{n=1}^{\infty}x^n$ converges for all $x \in (-1,1)$. 
        Because $0 \le |g(x)| \le \sum_{n=1}^{\infty}x^n$, we know that $g(x)$ converges absolutely.
        And due to the Absolute Conergence Test, we know that $\sum_{n=1}^{\infty}g(x)$ converges for all $x \in (-1,1)$
        Consequently, we see that $g(x)$ is defined for all $x \in (-1,1)$. 
        For when $x = 1$, we see that $g(1) = 1 - \frac{1}{2} + \frac{1}{3} - \frac{1}{4} + \dots$ is the alternating harmonic series, and we know that converges (simpily check that it fits the criteria for the Alternating Series Test). 
        And so from Abel's Test, we know that it thus converges uniformly on $[0,1]$. and also converges absolutely for any $x$ such that $|x|<1$, which implies it uniformly for $x \in (-1,1)$. 
        And using the Term-by-Term Continuity Theorem (we know that each partial sum is continous because it is a polynomial), we can say that $f$ is continous on $(-1,1]$. 
        Notice, this also implies that $g$ is continous on $(-1,1)$. 
        On $[-1,1]$, $g(x)$ is not even defined, because we see that $g(-1) = 1 + \frac{1}{2} + \frac{1}{3} + \dots$ is the harmoic series, whcih we know diverges. 
        And we also know that it is not possible for $g(x)$ to converg for any other points $|x|>1$. 
        As if we assume such point exists, then we know $|-1| < x$, and so $g(-1)$ converges, and we know that to be a contradiction, and we know that to be a contradiction. 
    }
\end{enumerate}

\subsection*{Problem 6}


\vspace*{1cm}


\section*{Section 6.6}
\textit{Problems: 2a}

\subsection*{Problem 2}
\begin{enumerate}[a).]
    \item {
    }
\end{enumerate}


\vspace*{1cm}


\section*{Section 7.2}
\textit{Problems: 4}

\subsection*{Problem 4}
We know that there exists a partition $P$ with $L(g, P) = U(g,P)$. 
$P$ can be expressed as ${x_0, x_1, \dots, x_n}$, and we see that 
\begin{align*}
&\sum_{k=1}^{n}m_k(x_k - x_{k-1}) = \sum_{k=1}^{n}M_k(x_k - x_{k-1}) \\
\implies &\sum_{k=1}^{n}(M_k - m_k)(x_k - x_{k-1}) = 0\\
\end{align*}
We know that $x_{k} - x_{k-1} \ne 0$, thus $M_k = m_k$.
This indicates that $\sup \{g(x): x \in [x_{k-1}, x_{k}]\}$ = $\inf \{g(x): x \in [x_{k-1}, x_{k}]\}$. 
We can easily see that $f(x)$ is a constant over the interval $[x_{k-1}, x_k]$ (A rather simple proof using contradiciton), and $g(x_{k-1}) = g(x_k)$.
And thus we see $g(x_0) = g(x_1) = \dots = g(x_n)$ (via induction), and so we know that $g(x)$ is constant over the entire interval [a,b]. 
This means it is integrable (constant implies continuity implies integrability). 
And the value of $\int_{a}^{b}g = (b-a)g(c)$ for any $c \in [a,b]$. 






\end{document}
