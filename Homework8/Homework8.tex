\documentclass[12pt]{article}

\usepackage{graphicx}			% Use this package to include images
\usepackage{amsmath}			% A library of many standard math expressions
\usepackage[margin=1in]{geometry}% Sets 1in margins. 
\usepackage{fancyhdr}			% Creates headers and footers
\usepackage{enumerate}          %These two package give custom labels to a list
\usepackage[shortlabels]{enumitem}
\usepackage{amsmath}
\usepackage{amssymb}
\usepackage{amsthm}
\usepackage{mathtools}

% Creates the header and footer. You can adjust the look and feel of these here.
\pagestyle{fancy}
\fancyhead[l]{Xiaoming Cao}
\fancyhead[c]{MATH521H Homework\#8}
\fancyhead[r]{\today}
\fancyfoot[c]{\thepage}
\renewcommand{\headrulewidth}{0.2pt} %Creates a horizontal line underneath the header
\setlength{\headheight}{15pt} %Sets enough space for the header

\newtheorem*{proposition}{Proposition}
\newtheorem{lemma}{Lemma}
\newtheorem{case}{Case}[]
\newtheorem*{claim}{Claim}


\setlength{\parindent}{0px}
\begin{document} %The writing for your homework should all come after this. 

\section*{Section 4.5}
\textit{Problems: 2, 6}
\subsection*{Problem 2}
\begin{enumerate}[a).]
    \item {
        We take $f(x) = sin(x)$ on the domain of $(-4\pi, 4 \pi)$, which is an open interval, and we see that the range is $[-1, 1]$, which is a closed interval.
    }
    \item {
        Impossible. 
        We know that a closed interval is a compact set (since it is closed and bounded), and we know that a continous function perserves compact sets. 
        Thus we know that the image of the function is compacted. 
        However, we know that an open interval is not compact, thus it cannot possibly be our range. 
    }
    \item {
        We see that 
        $$
            f(x) =
            \left\{
                \begin{array}{ll}
                    x^2 \quad -1 < x < 1 \\
                    tan(x-1)+1 \quad 1 \le x < 1 + \frac{\pi}{2}
                    
                \end{array} \right.
        $$
        And observe that the domain is $(-1, 1 + \frac{\pi}{2})$ with a range of $[0, \infty)$. 
    }
    \item {
        We see that between any $a,b \in \mathbb{Q}$, there exists $c \in \mathbb{I}$ such that $a< c < b$ due to the density the irrationals. 
        For some $x, y \in A$, where $A$ is the open interval which the function is defined on.
        And note that $[x,y] \subseteq A$.
        Without lost of generality, assume $f(x) < f(y)$, we know there exists some $f(x) < L < f(y)$ where $L \in \mathbb{I}$. 
        And because $f(x)$ is continous, thus by the mean value theorem, we know that there exists $c \in (x,y)$ such that $f(x) = L$. 
        Consequently, the range is not equal to $\mathbb{Q}$. 
    }
\end{enumerate}

\subsection*{Problem 6}
\begin{enumerate}[a).]
    \item {
        \begin{proof}
            We will let $g(x) \coloneq f(x) - f(x + \frac{1}{2})$. 
            And we see that $g(x)$ is contnous on $[0,\frac{1}{2}]$ since it is the sum of two functions continous on $[0,\frac{1}{2}]$.
            And observe that $g(0) = f(0) - f(\frac{1}{2})$ and $g(\frac{1}{2}) = f(\frac{1}{2}) - f(1) = -(f(0) - f(\frac{1}{2})) $. 
            We may consider three cases: $f(\frac{1}{2}) = f(0)$, $f(\frac{1}{2}) > f(0)$, and $f(\frac{1}{2}) < f(0)$. 
            If $f(0) = f(\frac{1}{2})$, we are done. However, if $f(0) < f(\frac{1}{2})$ or $f(0) > f(\frac{1}{2})$, we can exploit the mean value theorem. 
            Without lost of generality, assume $f(0) > f(\frac{1}{2})$. 
            We know that $g(0) > 0$ and $g(\frac{1}{2}) = - g(0) < 0$, thus $g(\frac{1}{2}) < 0 < g(0)$. 
            We know that $g(x)$ is continuous on $[0,\frac{1}{2}]$, consequently, through the intermediate value theorem, we know there exists a $c \in (0,\frac{1}{2})$ such that $g(c) = 0$.
            Which means $0 = f(c) - f(c+\frac{1}{2})$, and finally $f(c) = f(c+\frac{1}{2})$. Note $|c - (c+\frac{1}{2})| = \frac{1}{2}$.
            
        \end{proof}
    }
\end{enumerate}

\vspace*{1cm}


\section*{Section 5.2}
\textit{Problems: 2, 6, 9}
\subsection*{Problem 2}
\begin{enumerate}[a).]
    \item {
        We let
        $$ 
                f(x) = \left\{ 
                    \begin{array}{ll} 
                        1 \quad   x \ge 0\\
                        -1 \quad   x < 0\\
                    \end{array} \right.
        $$
        $$ 
                g(x) = |x|
        $$
        And observe that both not differentiable at $0$, and $f(x)g(x) = x$, which is differentiable at $0$. 
    }
    \item {
        We let 
        $$ 
                f(x) = \left\{ 
                    \begin{array}{ll} 
                        1 \quad   x \ge 0\\
                        -1 \quad   x < 0\\
                    \end{array} \right.
        $$
        $$
        g(x) = 0
        $$
        And observe that 
        $$ 
                f(x)g(x) = 0
        $$
        which is differentiable at $0$. 
    }
    \item {
        False. 
        We will use an infirect proof. 
        Observe that $f(x) + g(x)$ and $g(x)$ are differentiable, thus we know that $(f(x) + g(x)) - g(x) = f(x)$ is differentiable. 
        However, we know that $f(x)$ is not differentiable. 
        Consequently, we have arrived at our contradiction. 
    }
    \item {
        We let
        $$ 
                f(x) = \left\{ 
                    \begin{array}{ll} 
                        x^2 \quad   x \in \mathbb{Q}\\
                        0 \quad   x \in \mathbb{I}\\
                    \end{array} \right.
        $$
        Observe that the only continuous and differentiable point of the function is at $x = 0$. 
    }
\end{enumerate}


\subsection*{Problem 6}
\begin{enumerate}[a).]
    \item {
        We know that $g'(c) = \lim_{x \rightarrow c} \frac{g(x) - g(c)}{x - c}$, we let $h \coloneq x - c$, and observe that 
        $$g'(c) = \lim_{x \rightarrow c} \frac{g(x) - g(c)}{x - c} = \lim_{h \rightarrow 0} \frac{g(c+h) - g(c)}{h}$$
    }
    \item {
        Because $A$ is open, we know $c \in A$ cannot be one of the end points where the limit only exists from one-side. 
        $$g'(c) = \frac{1}{2} \lim_{h \rightarrow 0} (\frac{g(c+h) - g(c)}{h}+\frac{g(c) - g(c-h)}{h})$$
        It is easy to show that the second term is also an equivalent form of $g'(c)$ by using $\lim_{x\rightarrow c}\frac{g(c) - g(x)}{c-x}$ and $h \coloneq c-x$. 
        And we have
        $$g'(c) = \lim_{h \rightarrow 0} \frac{g(c+h) - g(c-h)}{2h}$$

    }
\end{enumerate}
\subsection*{Problem 9}
\begin{enumerate}[a).]
    \item {
        \begin{proof}
        Because we know $f'(x)$ exists on an interval $I$ and is not constant, we know for some $a, b \in I$, f(x) differentiable on [a,b], and $f(a) \ne f(b)$. 
        Without lost of generality, we let $f(b) > f(a)$, and because of the density of irrationals, we know between any two distinct real numbers(in this case $f(a)$ and $f(b)$), these exsits some $L \in \mathbb{I}$. 
        And using Daboux's theorem, we know that if $f(x)$ is differentiable on $[a,b]$, and $f'(a) < L < f'(b)$, then these exist a $c \in [a,b]$ such that $f'(c) = L$.
        Consequently, we have shown that $f'(x)$ must take on some irrational numbers. 
        \end{proof}
    }
    \item {
        False. 
        The counter example is the modified version of $x^2\sin(\frac{1}{x})$, that is $f(x) = x^2\sin(\frac{1}{x}) + \frac{1}{8}$. 
        Notice that at $x = 0$, $f'(0) = \frac{1}{8} > 0$. 
        However, at all $x \ne 0$, we see that $f'(x) = -\cos(\frac{1}{x}) + 2x\sin(\frac{1}{x}) + \frac{1}{8}$. 
        We define a sequence $(x_n) = \frac{1}{2\pi n}$ for $n \in \mathbb{N}$.
        We see that for all $\delta > 0$, there exists $x_n \in V_{\delta}(0)$ since $(\frac{1}{2\pi n}) \rightarrow 0$, and we know that for all $x_n$, $f'(x_n) = -cos(2 \pi n) + 2(\frac{1}{2 \pi n})sin(2\pi n) + \frac{1}{8} = -1 + \frac{1}{8} = -\frac{7}{8}$
        We see that for all $\delta$-neighborhood, $f'(x) < 0$ for some $x \in V_{\delta}(c)$. 
    }
\end{enumerate}


\vspace*{1cm}


\section*{Section 5.3}
\textit{Problems: 2}
\subsection*{Problem 2}
\begin{proof}
We will use an indirect proof. 
We assume for contradiction that $f'(x) \ne 0$ on $A$, and it is not the case that $f(x)$ is one-to-one on $A$. 
That is, we know there exist $a, b \in A$ such that $a \ne b$, but $f(x) = f(b)$. 
With out lost of generality, we take interval $[a,b]$, because $a, b \in A$, we know $[a,b] \subseteq A$, and thus this implies that $f(x)$ differentiable on $[a,b]$, which means it is also continuous. 
Using Rolle's Theorem, we know there exists a point $c \in (a,b)$ where $f'(c) = 0$ since $f(a) = f(b)$. 
And $c \in (a,b)$ imples $c \in A$.
Consequently, we have arrived at our contradiction because we assumed that $f'(x) \ne 0$ on $A$. 
An example to show that the converse statement need not be true if $f(x) = x^3$.
We see it is one-to-one on $\mathbb{R}$, but $f'(0) = 0$ and $0 \in \mathbb{R}$. 
\end{proof}


\end{document}
