\documentclass[12pt]{article}

\usepackage{graphicx}			% Use this package to include images
\usepackage{amsmath}			% A library of many standard math expressions
\usepackage[margin=1in]{geometry}% Sets 1in margins. 
\usepackage{fancyhdr}			% Creates headers and footers
\usepackage{enumerate}          %These two package give custom labels to a list
\usepackage[shortlabels]{enumitem}
\usepackage{amsmath}
\usepackage{amssymb}
\usepackage{amsthm}

% Creates the header and footer. You can adjust the look and feel of these here.
\pagestyle{fancy}
\fancyhead[l]{Xiaoming Cao}
\fancyhead[c]{MATH521H Homework\#6}
\fancyhead[r]{\today}
\fancyfoot[c]{\thepage}
\renewcommand{\headrulewidth}{0.2pt} %Creates a horizontal line underneath the header
\setlength{\headheight}{15pt} %Sets enough space for the header

\newtheorem*{proposition}{Proposition}
\newtheorem{lemma}{Lemma}
\newtheorem{case}{Case}[]
\newtheorem*{claim}{Claim}


\setlength{\parindent}{0px}
\begin{document} %The writing for your homework should all come after this. 

\section*{Section 3.3}
\textit{Problems: 1, 4, 5}
\subsection*{Problem 1}
\begin{proof}
    Because $K$ is compact, we know that it is closed and bounded. 
    We see that $K$ is bounded(both bounded above and below) and non-empty, so it contains its supremum and infimum. 
    For $\sup K$, the supremum of the set $K$, we know that there exists $k$ such that $\sup K - \epsilon < k < \sup K$. 
    And from which we can construct a sequence $(k_n)$ contained in $K$ that converge to $\sup K$.
    If $\lim_{}(k_n) \in K$, then we are done, so assuming that $\lim_{}(k_n) \notin K$, we will use contradiction.
    We know that $(k_n)$ is contained in $K$, converges, and $k_n \neq \lim_{}(k_n)$ for all $n \in \mathbb{N}$, we see that it is thus a limit point.
    We know that $K$ is closed, so it contains all its limit points, thus $\lim_{}(k_n) \in K$.
    Contradiction. 
    And thus we see that $\sup K \in K$. 
    The same arguemnt can be made for the infimum of $K$.
\end{proof}


\subsection*{Problem 4}
\begin{enumerate}[a).]
    \item {
        $K \cap F$ is definitely close and definitly compact. 
        We know that $K$ is compact, therefore it is closed and bounded, and we know that the intersection of two closed sets is closed, thus $K \cap F$ is closed.
        For any $x \in K \cap F$, we know that $x \in K$, and we know that $K$ is bounded, so $K \cap F$ is bounded. 
        And because it is closed and bounded, thus $K \cap F$ is compact.
    }
    \setcounter{enumi}{2}
    \item {
        It is not definitely closed or compact.
        We let $K = [0,1]$, we see it is compact, we let $F = \{1\}$, we see it is closed. 
        However, we see that $K - F = [0,1)$, and it is not closed since it does not contain the limit point $1$, and consequenctly, also not bounded. 
    }
\end{enumerate}
\subsection*{Problem 5}
\begin{enumerate}[a).]
    \item {
        True.
        \begin{proof}
            We know that the aribitrary intersection of closed set is closed. 
            And we also know that if one element is bounded, its intersection with other sets is also bounded. 
            And so the arbitrary intersection fo compact sets is both bounded and closed, thus compact.
        \end{proof}
    }
    \item {
        False. \\

        We will produce a counter-example. 
        We see that the set of $\{ n \}$, $n \in \mathbb{N}$ is closed and bounded, thus compact. 
        And we see that the arbitrary union of compact sets $\cup_{n=1}^{\infty}{n}$ is the set of natural numbers, and thus it is unbounded, so not compact.
    }
\end{enumerate}

\end{document}
