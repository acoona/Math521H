
\documentclass[12pt]{article}

\usepackage{graphicx}			% Use this package to include images
\usepackage{amsmath}			% A library of many standard math expressions
\usepackage[margin=1in]{geometry}% Sets 1in margins. 
\usepackage{fancyhdr}			% Creates headers and footers
\usepackage{enumerate}          %These two package give custom labels to a list
\usepackage[shortlabels]{enumitem}
\usepackage{amsmath}
\usepackage{amssymb}
\usepackage{amsthm}

% Creates the header and footer. You can adjust the look and feel of these here.
\pagestyle{fancy}
\fancyhead[l]{Xiaoming Cao}
\fancyhead[c]{MATH521H Homework\#2}
\fancyhead[r]{\today}
\fancyfoot[c]{\thepage}
\renewcommand{\headrulewidth}{0.2pt} %Creates a horizontal line underneath the header
\setlength{\headheight}{15pt} %Sets enough space for the header

\newtheorem*{proposition}{Proposition}
\newtheorem{lemma}{Lemma}
\newtheorem{case}{Case}[]
\newtheorem{claim}{Claim}[]


\setlength{\parindent}{0px}
\begin{document} %The writing for your homework should all come after this. 

\section*{Section 1.3}
\textit{Problems: 9}

\subsection*{Problem 9}
\begin{enumerate}[a).]
    \item {
    \begin{proof}
    We will use a direct proof. 
    Because $\sup A < \sup B$, we see $\sup B - \sup A > 0$.
    We know that $\sup A$ and $ \sup B \in \mathbb{R}$, thus there exists $n_0 \in \mathbb{N}$ such that $\frac{1}{n_0} < \sup B - \sup A$. 
    And we also know that for all $\epsilon > 0$, there exists $b \in B$ such that $b > \sup B - \epsilon$. 
    Let $\epsilon = \frac{1}{n_0}$, we see $b > \sup B - \frac{1}{n_0} > \sup A$ for some $b \in B$. 
    Consequently, we have shown that there exists an element $b \in B$ that is an upper bound for $A$ if $\sup B > \sup A$. 
    \end{proof}
    }

    \item {
    $A = \left( 0,1 \right)$\\
    $B = \left( 0, 1\right)$\\
    $\sup A = \sup B$
    *I might need more explaination here.

    }

\end{enumerate}



\vspace*{1cm}


\section*{Section 1.4}
\textit{Problems: 1, 4, 5, 8}

\subsection*{Problem 1}
\begin{enumerate}[a).]
    \item {
    \begin{proof}
        If $a, b \in \mathbb{Q}$, by definition, $a =\frac{p}{q}$ and $b = \frac{s}{t}$ for $p,q,s,t \in \mathbb{Z}$ and $q,t \ne 0$. 
        We will first show that $ab \in \mathbb{Q}$.
        We see
        \begin{align*}
            &ab\\
            =&(\frac{p}{q})(\frac{s}{t})\\
            =&\frac{ps}{qt}
        \end{align*}
        It is clear that $ps, qt\in \mathbb{Z}$ and $qt \ne 0$. 
        Consequently, we see that $ab \in \mathbb{Q}$
        We will now show that $a+b \in \mathbb{Q}$.
        We see
        \begin{align*}
            &a+b\\
            =&(\frac{p}{q})+(\frac{s}{t})\\
            =&\frac{pt+qs}{qt}
        \end{align*}
        One can easily see that $pt+qs, qt\in \mathbb{Z}$ and $qt \ne 0$. 
        Consequently, we see that $a+b \in \mathbb{Q}$
        
    \end{proof}
    }
    
    \item {
        \begin{proof}
            We will use contradiction to show that if $a \in \mathbb{Q}$ and $t \in \mathbb{I}$, then $a+t \in \mathbb{I}$. 
            Assume not, that is, $a \in \mathbb{Q}$, $t \in \mathbb{I}$, and $a+t \in \mathbb{Q}$. 
            Because $a+t \in \mathbb{Q}$, $a+t = \frac{p}{q}$ for $p,q \in \mathbb{Z}$ and $q \ne 0$. 
            In addition, because $a \in \mathbb{Q}$, $a = \frac{c}{d}$ for $c, d \in \mathbb{Z}$ and $d \ne 0$. 
            We see 
            \begin{align*}
                a + t &= a + t\\
                \frac{c}{d} + t &= \frac{p}{q}\\
                t &= \frac{p}{q}+\frac{c}{d}\\
                t &= \frac{pd+qc}{qd}\\
            \end{align*}
            Because $pd + cq, pd \in \mathbb{Z}$ adn $pd \ne 0$, we know that $t \in \mathbb{Q}$.
            Thus we have arrived at our contradiction because we know that $t \in \mathbb{R}$. 
        \end{proof}

        \begin{proof}
            We will use contradiction to show that if $a \in \mathbb{Q}$ and $t \in \mathbb{I}$, then $at \in \mathbb{I}$. 
            Assume not, that is, $a \in \mathbb{Q}$, $t \in \mathbb{I}$, and $at \in \mathbb{Q}$. 
            Because $at \in \mathbb{Q}$, $at = \frac{p}{q}$ for $p,q \in \mathbb{Z}$ and $q \ne 0$. 
            In addition, because $a \in \mathbb{Q}$, $a = \frac{c}{d}$ for $c, d \in \mathbb{Z}$ and $d \ne 0$. 
            We see 
            \begin{align*}
                at &= at\\
                \frac{c}{d}t &= \frac{p}{q}\\
                t &= \frac{pc}{qd}\\
            \end{align*}
            Because $pc, pd \in \mathbb{Z}$ and $pd \ne 0$, we know that $t \in \mathbb{Q}$.
            Thus we have arrived at our contradiction because we know that $t \in \mathbb{R}$. 
        \end{proof}
    }
    \item {
        $\mathbb{I}$ is not closed under addition and multiplication
        e.g. we see $\sqrt[]{2} \times \sqrt[]{2} = 2$, and $2 \in \mathbb{Q}$. 
        We do not know whether $s+t$ or $s+t$ will be a rational number or irrational number. 
    }
\end{enumerate}

\subsection*{Problem 4}
\begin{proof}
    We will first show that $b$ is an uppper bound.
    We know that for all $x \in \mathbb{Q} \cap [a,b]$, $a \le x \le b$, consequently, $b$ is an upper bound of $\mathbb{Q} \cap [a,b]$. 
    Now we will use a proof by contradiction to show that for any upper bound $c$, $b \le c$. 
    Assume there exists upper bound $c$ such that $c < b$. 
    We know that there exists $r \in \mathbb{Q}$ such that $c < r < b$ because of the rational density theorem. 
    We have $$a < c < x < b$$
    However, we can see $x \in [a,b] \cap \mathbb{Q}$, and so $c$ is not a upper bound. 
    Consequently, we have arrived at our contradiction, and know that for any upper bound $d$, $b\le d$. 
    Thus we have shown that $b$ is the least upper bound.
    
\end{proof}

\subsection*{Problem 5}
\begin{proof}
    We know any $c \in \mathbb{R}$ can be expressed as $c = a - \sqrt{2}$, where $a \in \mathbb{R}$. 
    And for any $d \in \mathbb{R}$, $d = b - \sqrt{2}$ for some $b \in \mathbb{R}$. 
    In addtion, we know that for some $s \in \mathbb{Q}$, if $t = s - \sqrt{2}$, then $t \in \mathbb{I}$. 
    We let $ c < d$, and we want to show that
    \begin{align*}
        c &< t < d\\
        \implies a - \sqrt{2} &< s - \sqrt{2} < b - \sqrt{2}\\
        \implies a &< s < b 
    \end{align*}
    That is, for some $a,b \in \mathbb{R}$, $a < b$, there exists $s \in \mathbb{Q}$ such that $a < s < b$ (this is theorem 1.4.3 of the text book). 
    Because of the Archimedean property, we know that there exists $n \in \mathbb{N}$ such that $\frac{1}{n} < b - a$. 
    And because 

    

    
    
\end{proof}

\subsection*{Problem 8}
\begin{enumerate}[a).]
    \item {
    $A = \{ \frac{1}{2n}, n \in \mathbb{N}\}$\\
    $B = \{ \frac{1}{2n+1}, n \in \mathbb{N}\}$
    *I might need more explaination here.
    }
\end{enumerate}

\end{document}
