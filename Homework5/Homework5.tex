\documentclass[12pt]{article}

\usepackage{graphicx}			% Use this package to include images
\usepackage{amsmath}			% A library of many standard math expressions
\usepackage[margin=1in]{geometry}% Sets 1in margins. 
\usepackage{fancyhdr}			% Creates headers and footers
\usepackage{enumerate}          %These two package give custom labels to a list
\usepackage[shortlabels]{enumitem}
\usepackage{amsmath}
\usepackage{amssymb}
\usepackage{amsthm}

% Creates the header and footer. You can adjust the look and feel of these here.
\pagestyle{fancy}
\fancyhead[l]{Xiaoming Cao}
\fancyhead[c]{MATH521H Homework\#5}
\fancyhead[r]{\today}
\fancyfoot[c]{\thepage}
\renewcommand{\headrulewidth}{0.2pt} %Creates a horizontal line underneath the header
\setlength{\headheight}{15pt} %Sets enough space for the header

\newtheorem*{proposition}{Proposition}
\newtheorem{lemma}{Lemma}
\newtheorem{case}{Case}[]
\newtheorem*{claim}{Claim}


\setlength{\parindent}{0px}
\begin{document} %The writing for your homework should all come after this. 

\section*{Section 2.5}
\textit{Problems: 5}
\subsection*{Problem 5}
\begin{proof}
    We will use a proof by contradiction. 
    Assume that it is not the case that $\lim_{} = a$. 
    That is there exists $\epsilon > 0$, such that for all $N \in \mathbb{N}$, $n \ge N$ and $|a_n - a| \ge \epsilon$. 
    And thus we can construct a subsequence of $a_n$ such that $|a_{n_k} - a| \ge \epsilon$ . 
    We know that $(a_n)$ is bounded, so $(a_{n_k})$ is bounded, since it is a subsequence of $(a_n)$. 
    And from the Bolzano-Weierstrass theorem, we know that $(a_{n_k})$ contains a converging subsequence $(a_{n_{k_j}})$. 
    And we know that there exists $\epsilon > 0$ such that for all $n \in \mathbb{N}$, $|a_{n_{k_{k}}} - a| \ge \epsilon$, therefore the subsequence does not converge to $a$. 
    However, we see that $(a_{n_{k_j}})$ is all so a subsequence of $a_n$, so by assumption, it converges to $a$. 
    And consequently, we have arrived at our contradiction, and thus establishing the validity of our original statement. 
\end{proof}


\vspace*{1cm}


\section*{Section 2.6}
\textit{Problems: 2}
\subsection*{Problem 2}
\begin{enumerate}[a).]
    \item {
        We see that $(a_n) = (1, -\frac{1}{2}, \frac{1}{3}, -\frac{1}{4}, \dots, (-1)^{n+1}(\frac{1}{n}))$ is Cauchy, but it is not monotone ($a_1 > a_2$ and $a_2 < a_3$). 
        Hence, we have found our example.
    }
    \item {
        Impossible, since we know that all Cauchy sequence is convergent, and all the subsequences of a convergent sequence is also convergent, and thus all convergent sequences are bounded, and so all the subsequences are bounded and an unbounded subsequence does not exists. 
    }
    \item {
        Impossible.
        We name the squence $(a_n)$, and we know that it is a divergent monotone sequence. 
        Hence it is not bounded, in addition, we know that $|a_n| \le |a_{n+1}|$ (It would be bounded otherwise). 
        Because $a_n$ it is not bounded, $\forall M \in \mathbb{R}$, $\exists N \in \mathbb{N}$, such that $|a_N| \ge M$. 
        And we see that $\forall n \ge N$, $|a_{n}| \ge M$. 
        consequently, we see that all the subsequences of $(a_n)$ is unbounded (when $k \ge N$, $n_k \ge k \ge N$). 
        However, we know that there exists a Cauchy subsequence, hence it is convergent, hence it is bounded. 
        But we know that all subsequenes are unbounded, therefore we have arrived at our contradiction.
    }
    \item {
        $(a_n) = (1, 0, 2, 0, 4, 0, 5, \dots )$ has subsequence $(0,0,0, \dots)$ which is Cauchy. 
    }
\end{enumerate}


\vspace*{1cm}


\section*{Section 2.7}
\textit{Problems: 4, 7a}
\subsection*{Problem 4}
\begin{enumerate}[a).]
    \item {
        $\sum x_n = 1 + \frac{1}{2} + \frac{1}{3} + \dots + \frac{1}{n}$ diverges\\

        $\sum y_n = 1 + \frac{1}{2} + \frac{1}{3} + \dots + \frac{1}{n}$ diverges\\

        We see that 
        $$\sum x_ny_n = 1 + \frac{1}{4} + \frac{1}{9} + \dots + \frac{1}{n^2}$$\\
        which converges(in the form $\frac{1}{n^p}$, where $p > 1$).
    }
    \item {
        We know that $\sum x_n = 1 - \frac{1}{2} + \frac{1}{3} - \dots + (-1)^{n+1}(\frac{1}{n})$ converges.
        And we take a bounded sequence $(y_n) = (1, -1, 1, \dots, (-1)^n+1)$. 
        We see that $\sum x_ny_n = 1 + \frac{1}{2} + \frac{1}{3} + \dots + \frac{1}{n}$, which diverges.

    }
    \item {
        The request is impossible.
        Since $\sum (x_n + y_n)$ and $\sum x_n$ converges, using the algebraic limit theorem, we know that $\sum(x_n + y_n) - \sum x_n = \sum y_n$ converges.
        But we know that $\sum y_n$ diverges, hence we have arrived at our contradiction.
    }
    \item {
        We can let $x_n = (0, \frac{1}{2}, 0, \frac{1}{4}, 0, \frac{1}{6}, \dots)$. 
        We see that $\sum (-1)^n x_n = \sum \frac{1}{2n} = 2 \sum \frac{1}{n}$. 
        We konw that $\sum \frac{1}{n} = (2)(\frac{1}{2n})$ diverges, thus $2 \sum \frac{1}{n}$ diverges.
    }
\end{enumerate}


\subsection*{Problem 7}
\begin{enumerate}[a).]
    \item {
        \begin{proof}
            Because $\lim (n a_n) = l$, we know that $l > 0$ (The proof is trivial employing conradiction and the definition is limit, assume $l < 0$ and set $\epsilon = \frac{|l|}{2}$, show $a$ is negative, hence contradiction).
            $\forall \epsilon > 0$, $N \in \mathbb{N}$, $ n \ge N \implies |n(a_n) - l | < \epsilon$. 
            We can set $\epsilon = \frac{l}{2}$, we see that $\frac{l}{2} < n(a_n) < \frac{3l}{2}$. 
            And we see that $\frac{l}{2n} < a_n$. 
            Because $\sum \frac{l}{2n} = \frac{l}{2}\sum\frac{1}{n}$, and we know the harmonic series diverges, so it also diverges. 
            And we know $\frac{l}{2n} < a_n$ and $0 < \frac{l}{2n} < a_n$ so by the comparison test, we know that $\sum a_n$ diverges.
        \end{proof}
    }
\end{enumerate}


\vspace*{1cm}


\section*{Section 3.2}
\textit{Problems: 2, 4a, 8ab, 11a}
\subsection*{Problem 2}
\begin{enumerate}[a).]
    \item {
        limit points: \\
        For set $A$: $-1$ and $1$. \\
        For set $B$: $[0,1]$. 
    }
    \item {
        Both sets are not closed, as they both don't contain all of their limit points. 
        We see that set $A$ doesn't contain $-1$, which is a limit point. 
        And we see that for all $\epsilon > 0$, $V_e(\frac{\sqrt{2}}{2}) \cap B \ne \emptyset$ or $\frac{\sqrt{2}}{2}$ (rationals are dense), thus $\frac{\sqrt{2}}{2}$ is a limit point not contained. 
        And we see that both $A$ and $B$ are not open, because we know that irrationals are dense, and so irrationals are contained in any $V_{\epsilon}(x)$ for $A$. 
        consequently, causing $V_{\epsilon} (x)$ to not be a subset of $A$. 
        The reasoning for $B$ is the same.
    }
    \item {
        For set $A$, we see that all points except for $1$ is an isolated point.
        Because we know that for any other arbirary point, $a_n$, $\exists \epsilon > 0$ such that $V_{\epsilon}(a) \cap A = \emptyset$, namely, let $\epsilon = \frac{1}{n} - \frac{1}{n + 2}$.
        We see that $B$ does not contain any limit point, as the rationals are also dense, hence in any $\epsilon$ interval, there exists rationals. 

    }
    \item {
        We see that $\bar{A} = A \cup \{-1\}$, and we see that $\bar{B} = \{x \in \mathbb{R} : 0 \le x \le 1\}$

    }
\end{enumerate}



\subsection*{Problem 4}
\begin{enumerate}[a).]
    \item {
        \begin{proof}
            We see that if $s = a$ for some $a \in A$, then we see that $s \in A \subseteq \bar{A}$, thus we are done.
            And for $s \ne a$ for some $a \in A$, we know that for all $\epsilon > 0$, $s - \epsilon < a_1 < s < s + \epsilon$ for some $a_1 \in A$. 
            We see that for all $\epsilon > 0$, $V_{\epsilon}(s) \cap A \ne \emptyset$ or equal to $s$ (since $s \notin A$). 
            Consequently, $s$ is a limit point of $A$, $s \in L$, and thus $s \in \bar{A}$. 
        \end{proof}
    }
\end{enumerate}

\subsection*{Problem 8}
We know that the only sets that are both open and close are $\mathbb{R}$ and $\emptyset$. 

\begin{enumerate}[a).]
    \item {
        We know that the closure of any set, in this case $ A \cup B$, is closed. 
        Thus, definitly closed
    }

    \item {
        We know that $A - B$ is equivalent to $A \cap B^\complement$.
        And because $B$ is closed, so we know that $B^\complement$ is open. 
        The interseciton of an finite number of open sets is open. 
        Thus, we see that $A - B$ is open. 
    }
\end{enumerate}

\subsection*{Problem 11}
\begin{enumerate}[a).]
    \item {
        \begin{proof}
            We will show that $\overline{A \cup B} \subseteq \bar{A} \cup \bar{B}$ and $\overline{A \cup B} \supseteq \bar{A} \cup \bar{B}$, thus they are equivalent. 
            We know that $\overline{A} = A \cup L_A$, and same for $B$. 
            So we see that $A \cup B \subseteq \overline{A} \cup \overline{B}$ (unions of sets are commutative). 
            And it is easy to see from definition that $\overline{A \cup B} \subseteq \overline{\overline{A} \cup \overline{B}}$ ($\forall \epsilon > 0$, $V_{\epsilon}(x) \cap (A \cup B) \ne \{x\}$ and not empty $\implies$ $V_{\epsilon}(x) \cap (\overline{A} \cup \overline{B}) \ne \{x\}$ and not empty). 
            We know that the union of two close set is a closed set, thus $\overline{A} \cup \overline{B}$ is a close set, and so $\overline{\overline{A} \cup \overline{B}} = \overline{A} \cup \overline{B}$.
            Thus, $\overline{A \cup B} \subseteq \bar{A} \cup \bar{B}$. 
            To show that the other way is true, we see that $\overline{A} \subseteq \overline{A \cup B}$, and $\overline{B} \subseteq \overline{A \cup B}$ (can be easily seen through definition of limit point, like above).
            Thus, $\overline{A} \cup \overline{B} \subseteq \overline{A \cup B}$.  
            And finally we have shown that they are equivalent. 
            
        \end{proof}
    }
\end{enumerate}

\end{document}
