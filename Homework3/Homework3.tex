\documentclass[12pt]{article}

\usepackage{graphicx}			% Use this package to include images
\usepackage{amsmath}			% A library of many standard math expressions
\usepackage[margin=1in]{geometry}% Sets 1in margins. 
\usepackage{fancyhdr}			% Creates headers and footers
\usepackage{enumerate}          %These two package give custom labels to a list
\usepackage[shortlabels]{enumitem}
\usepackage{amsmath}
\usepackage{amssymb}
\usepackage{amsthm}

% Creates the header and footer. You can adjust the look and feel of these here.
\pagestyle{fancy}
\fancyhead[l]{Xiaoming Cao}
\fancyhead[c]{MATH521H Homework\#3}
\fancyhead[r]{\today}
\fancyfoot[c]{\thepage}
\renewcommand{\headrulewidth}{0.2pt} %Creates a horizontal line underneath the header
\setlength{\headheight}{15pt} %Sets enough space for the header

\newtheorem*{proposition}{Proposition}
\newtheorem{lemma}{Lemma}
\newtheorem{case}{Case}[]
\newtheorem{claim}{Claim}[]


\setlength{\parindent}{0px}
\begin{document} %The writing for your homework should all come after this. 

\section*{Section 1.5}
\textit{Problems: 4abc, 5, 6ab, 10a}
\subsection*{Problem 4}
\begin{enumerate}[a).]
    \item {
We see that $f(x) = \frac{x}{x^2 - 1}$ has derivative $f'(x) = - \frac{1 + 2x^2}{(x^2 - 1)^2}$.
And for all $x \in (-1, 1)$ we see that $f'(x) < 0$. 
Conseqeuntly, $f(x)$ doesn't "loop back" (for $a,b \in (-1,1)$, if $a < b$, then $f(a) > f(b)$ ), and hence it is injective. 
We will now show that it is surjective. 
We see $\lim_{x \rightarrow -1^+}f(x) = + \infty$ and $\lim_{x \rightarrow 1^-}f(x) = - \infty$, and invoking the intermediate value theorem, we can see that it has a range of $(- \infty, \infty)$, consequently surjective. 
We see the function $g(x) = \frac{x - \frac{a - b}{2}}{(\frac{a-b}{2})}$ describes merely a shift along with a scaling from $(a,b)$ to $(-1,1)$, thus it is bijective.
And $h(x) = f(g(x))$ is a bejective function(composition of bijective functions is bejiective) which shows $(a,b) \sim (- \infty, \infty)$. 
    }
    \item {
        We let $g(x) = \frac{2}{x-a + 1} - 1$ for $x \in (a, \infty)$. 
        We will first show that the function is injective using contradiction. 
        For $x_1 \ne x_2$, we assume $f(x_1) = f(x_2)$, we see
        \begin{align*}
            \frac{2}{x_1-a + 1} - 1 &= \frac{2}{x_2-a + 1} - 1 \\
            \implies \frac{2}{x_1-a + 1} &= \frac{2}{x_2-a + 1} \\
            \implies x_1 &= x_2
        \end{align*}
        Conseqeuntly, $g(x)$ is injective. 
        We see that $g(x)$ is continuous for $x \in (a, \infty)$, and $\lim_{x \rightarrow a^+}g(x) = 1$ and $\lim_{x \rightarrow \infty}g(x) = -1$. 
        Thus, using the intermediate value, we know that $x$ is surjective to the co-domain $(-1,1)$. 
        We know that the composition of two bijective functions is bijective, thus, we we define $p(x) = g(h(x))$ using $h(x)$ from part a). 
        Finally this shows that $(a, \infty) \sim \mathbb{R}$. 
    }
    \item {
        $ 
        f(x) = \left\{ 
            \begin{array}{ll} 
                \frac{1}{2} \quad x = 0 \\
                \frac{x}{2} \quad x = \frac{1}{2^n} \; \text{for} \; n \in \mathbb{N} \\
                x \quad \text{else} 
            \end{array} \right.
        $

        We see that in the case when $f(x) = \frac{1}{2}$, the only solution is $x = 0$.
        For $f(x) = \frac{1}{2^n}$, where $n \in \mathbb{N}$, $n \ge 2$, the only solution is $x = \frac{1}{2^{n-1}}$. 
        And for all $ f(x) \ne \frac{1}{2^n}$ for some $n \in \mathbb{N}$, the only solution is $x = x$. 
        Conseqeuntly, we have show that f(x) have exactly one solution for all $x \in (0,1)$, thus it is bijective, and so $[0,1) \sim (0,1)$. 
    }

\end{enumerate}


\subsection*{Problem 5}
\begin{enumerate}[a).]
    \item {
        For every $x \in A$, $f(x) = x$ is a bijiection from $A$ to $A$. 
        Conseqeuntly, $A \sim A$. 
    }
    \item {
        Because $A \sim B$, we know that there exists a bijective function $f(x)$ such that $f: A \rightarrow B$. 
        And thus $f^{-1}(x)$ is also bijective and $f^{-1}: B \rightarrow A$. 
        Conseqeuntly, $B \sim A$
    }
    \item {
        Because $A \sim B$, we know that there exists a bijective function $f(x)$ such that $f: A \rightarrow B$. 
        From $B \sim C$, we know that there exists a bijective function $g(x)$ such that $g: B \rightarrow C$. 
        The composition of two bijective functions with the co-domain of the first equivalent to the domain of the second is also bijective. 
        Thus the bijective function $h(x) = f(g(x))$ has $h: A \rightarrow C$. Conseqeuntly, we have $A \sim C$. 

    }
\end{enumerate}


\subsection*{Problem 6}
\begin{enumerate}[a).]
    \item {
    $I_1 = (1, 2)$,
    $I_2 = (2, 3)$,
    and $I_n = (n, n+1)$, for $n \in \mathbb{N}$

    }
    \item {
        We let $M$ be an uncountable set, and let $I_m$ for $m \in M$ to be the collection of uncountable disjoint intervals.
        Namely, $I_m = (a_m, b_m)$. From the theorem of the density of the rational number which was proven in chapter 1, we know that between any two $a,b \in \mathbb{R}$, there exists a $c \in \mathbb{Q}$ such that $a < c < b$. 
        We can thus rename this set as $I_c$ for some $c \in \mathbb{Q}$. 
        The sole thing left to be established is the injectivity between $c \in \mathbb{Q}$ and $(a_m, b_m)$. 
        We know that 

    }
\end{enumerate}
\subsection*{Problem 10}


\vspace*{1cm}


\section*{Section 2.2}
\textit{Problems: 1, 2b}
\subsection*{Problem 1}
$a_n = (1,1,1,1,1,1, \dots)$, we see that $|a_n - 0| < 2$.
Because there exists $\epsilon > 0$ ($\epsilon = 2$ in this case) such that for all $N \in \mathbb{N}$, $n \ge N$, thus $a_n$ verconges to $0$. 
We know that the sequence $a_n = (1,-1,1,-1, \dots, (-1)^{n})$ diverges, but we see that $|a_n - 0| < 2$, thus it also verconges to 0.


\subsection*{Problem 2}
\begin{proof}
    For all $\epsilon > 0$, we choose $N$ such that $N > \frac{2}{\epsilon}$.
    For $n \ge N$, we see 
    \begin{align*}
        n &> \frac{2}{\epsilon} \\
        \frac{2}{n} &< \epsilon \\
        \frac{2n^2}{n^3} &< \epsilon \\
        \frac{2n^2}{n^3 + 3} &< \frac{2n^2}{n^3} < \epsilon \\
        \left| {\frac{2n^2}{n^3 + 3} - 0} \right| &< \epsilon
    \end{align*}
    Consequently, we see that $\lim_{n \rightarrow \infty} = 0$. 
\end{proof}


\end{document}
